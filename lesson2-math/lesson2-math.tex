\documentclass[11pt]{article}
\pagestyle{empty} %this removes page number%
\usepackage{amsmath, amssymb, amsfonts}

\begin{document}

superscripts
$$2x^3$$
$$2x^{34}$$
$$2x^{3x+4}$$

subscripts
$$x_1$$
$$x_{12}$$
$$x_{1_2}$$
$$x_{1_{2_3}}$$
$$a_0, a_1, a_2, \ldots, a_{100}$$

Greek letters
$$\pi$$
$$\Pi$$
$$\alpha$$
$$A=\pi r^2$$
$$y=\sin x$$
$$y=\cos x$$
$$y=\csc \theta$$
$$y=\sin^{-1} x$$
$$y=\arcsin x$$

Log functions
$$y=\log x$$
$$y=\log_5 x$$
$$y=\ln x$$

Roots

$$\sqrt{2}$$
$$\sqrt[3]{2}$$
$$\sqrt{x^2+y^2}$$
$$\sqrt{ 1+\sqrt{x} }$$

Fractions
$$\frac{2}{3}$$
About $\displaystyle \frac{2}{3}$ of the glass is full.\\[16pt] %without package - notice the display style makes two thirds larger%
About $\frac{2}{3}$ of the glass is full.\\[6pt] %small 2/3%
About $\dfrac{2}{3}$ of the glass is full. %with package%

$$\frac{\sqrt{x+1}}{\sqrt{x+2}}$$

$$\frac{1}{1 + \frac{1}{x}}$$

\section{Circle Question}

$$c = 2 \pi r$$

Circle A has radius 1.

Circle B has radius 3.

$\therefore$
$$c_A = 2 \pi r_A$$
$$c_B = 2 \pi r_B$$
Where $3r_A = r_B$\\
Hence 
$$c_B = 6 \pi r_A$$
$$c_B = 3 c_A$$

$$c_B + \frac{c_A}{2} = 6 \pi r_A + \pi r_A$$
$$c_B + \frac{c_A}{2} = 7 \pi r_A $$

Let $r_C = r_A + r_B = 4 r_A$

$$c_C = 2 \pi r_C = 8 \pi r_A$$

$$\frac{c_C}{c_A} = \frac{8 \pi r_A}{2 \pi r_A} = 4$$

$\therefore$ there are 4 revolutions in total



\end{document}