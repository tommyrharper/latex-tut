\documentclass[11pt]{article}
\usepackage{amsfonts, amssymb, amsmath}
\usepackage{float}
\parindent 0px %do not indent paragraphs%
\pagestyle{empty}

\begin{document}

The distributive property states that $a(b+c)=ab+ac$, for all $a, b, c \in \mathbb{R}$. \\[6pt]
The equivalence class of $a$ is $[a]$. \\[6pt]
The set $A$ is defined to be $\{1, 2, 3\}$. \\[6pt]
The movie tickets costs $\$11.50$. \\[6pt]

Notice how the bracktes are too small:
$$2(\frac{1}{x^2-1})$$
We can fix that using slash left and slash right:
$$2\left(\frac{1}{x^2-1}\right)$$
$$2\left[\frac{1}{x^2-1}\right]$$
$$2\left\{\frac{1}{x^2-1}\right\}$$
$$2\left \langle \frac{1}{x^2-1}\right \rangle$$
$$2\left | \frac{1}{x^2-1}\right |$$

$$\left.\frac{dy}{dx}\right|_{x=1}$$

$$\left( \frac{1}{1+\left(\frac{1}{1+x}\right)} \right)$$

Tables:\\

\begin{tabular}{|c||c|c|c|c|c|} %c standards for centered for a given column, you can also do l for left aligned and r for right aligned%
\hline
$x$ & 1 & 2 & 3 & 4 & 5 \\ \hline
$fx(x)$ & 10 & 11 & 12 & 13 & 14 \\ \hline
\end{tabular}

\vspace{1mm}

\begin{table}[H] %The H with the float package fixes this table in place%
\centering
\def\arraystretch{1.5} %add extra padding in table cells
\begin{tabular}{|c||c|c|c|c|c|} %c standards for centered for a given column, you can also do l for left aligned and r for right aligned%
\hline
$x$ & 1 & 2 & 3 & 4 & 5 \\ \hline
$fx(x)$ & $\frac{1}{2}$ & 11 & 12 & 13 & 14 \\ \hline
\end{tabular}
\caption{These values represent the function $f(x)$.}
\end{table}


\begin{table}[H]
\centering
\caption{The relationship between $f(x)$ and $f'(x)$.}
\def\arraystretch{1.5} %add extra padding in table cells
\begin{tabular}{|l|p{4in}|} %p is for paragraph at 4 inches wide%
\hline
$f(x)$ & $f'(x)$ \\ \hline
$ x>0$ & The function $f(x)$ is increasing. The function $f(x)$ is increasing. The function $f(x)$ is increasing. \\ \hline
\end{tabular}
\end{table}

Arrays:
\begin{align}
% \, forces a space in math mode
5x^2\, \text{place your words here}\\
5x^2-9=x+3\\
5x^2-x-12=0
\end{align}

\begin{align*}
5x^2-9&=x+3\\
5x^2-x-12&=0\\
&= 12 + x - 5x^2
\end{align*}

\begin{align}
    % &= aligns the equals signs
5x^2-9&=x+3\\
5x^2-x-12&=0\\
&= 12 + x - 5x^2
\end{align}

\end{document}
